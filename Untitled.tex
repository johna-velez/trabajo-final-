\documentclass[11pt]{article}

    \usepackage[breakable]{tcolorbox}
    \usepackage{parskip} % Stop auto-indenting (to mimic markdown behaviour)
    
    \usepackage{iftex}
    \ifPDFTeX
    	\usepackage[T1]{fontenc}
    	\usepackage{mathpazo}
    \else
    	\usepackage{fontspec}
    \fi

    % Basic figure setup, for now with no caption control since it's done
    % automatically by Pandoc (which extracts ![](path) syntax from Markdown).
    \usepackage{graphicx}
    % Maintain compatibility with old templates. Remove in nbconvert 6.0
    \let\Oldincludegraphics\includegraphics
    % Ensure that by default, figures have no caption (until we provide a
    % proper Figure object with a Caption API and a way to capture that
    % in the conversion process - todo).
    \usepackage{caption}
    \DeclareCaptionFormat{nocaption}{}
    \captionsetup{format=nocaption,aboveskip=0pt,belowskip=0pt}

    \usepackage[Export]{adjustbox} % Used to constrain images to a maximum size
    \adjustboxset{max size={0.9\linewidth}{0.9\paperheight}}
    \usepackage{float}
    \floatplacement{figure}{H} % forces figures to be placed at the correct location
    \usepackage{xcolor} % Allow colors to be defined
    \usepackage{enumerate} % Needed for markdown enumerations to work
    \usepackage{geometry} % Used to adjust the document margins
    \usepackage{amsmath} % Equations
    \usepackage{amssymb} % Equations
    \usepackage{textcomp} % defines textquotesingle
    % Hack from http://tex.stackexchange.com/a/47451/13684:
    \AtBeginDocument{%
        \def\PYZsq{\textquotesingle}% Upright quotes in Pygmentized code
    }
    \usepackage{upquote} % Upright quotes for verbatim code
    \usepackage{eurosym} % defines \euro
    \usepackage[mathletters]{ucs} % Extended unicode (utf-8) support
    \usepackage{fancyvrb} % verbatim replacement that allows latex
    \usepackage{grffile} % extends the file name processing of package graphics 
                         % to support a larger range
    \makeatletter % fix for grffile with XeLaTeX
    \def\Gread@@xetex#1{%
      \IfFileExists{"\Gin@base".bb}%
      {\Gread@eps{\Gin@base.bb}}%
      {\Gread@@xetex@aux#1}%
    }
    \makeatother

    % The hyperref package gives us a pdf with properly built
    % internal navigation ('pdf bookmarks' for the table of contents,
    % internal cross-reference links, web links for URLs, etc.)
    \usepackage{hyperref}
    % The default LaTeX title has an obnoxious amount of whitespace. By default,
    % titling removes some of it. It also provides customization options.
    \usepackage{titling}
    \usepackage{longtable} % longtable support required by pandoc >1.10
    \usepackage{booktabs}  % table support for pandoc > 1.12.2
    \usepackage[inline]{enumitem} % IRkernel/repr support (it uses the enumerate* environment)
    \usepackage[normalem]{ulem} % ulem is needed to support strikethroughs (\sout)
                                % normalem makes italics be italics, not underlines
    \usepackage{mathrsfs}
    

    
    % Colors for the hyperref package
    \definecolor{urlcolor}{rgb}{0,.145,.698}
    \definecolor{linkcolor}{rgb}{.71,0.21,0.01}
    \definecolor{citecolor}{rgb}{.12,.54,.11}

    % ANSI colors
    \definecolor{ansi-black}{HTML}{3E424D}
    \definecolor{ansi-black-intense}{HTML}{282C36}
    \definecolor{ansi-red}{HTML}{E75C58}
    \definecolor{ansi-red-intense}{HTML}{B22B31}
    \definecolor{ansi-green}{HTML}{00A250}
    \definecolor{ansi-green-intense}{HTML}{007427}
    \definecolor{ansi-yellow}{HTML}{DDB62B}
    \definecolor{ansi-yellow-intense}{HTML}{B27D12}
    \definecolor{ansi-blue}{HTML}{208FFB}
    \definecolor{ansi-blue-intense}{HTML}{0065CA}
    \definecolor{ansi-magenta}{HTML}{D160C4}
    \definecolor{ansi-magenta-intense}{HTML}{A03196}
    \definecolor{ansi-cyan}{HTML}{60C6C8}
    \definecolor{ansi-cyan-intense}{HTML}{258F8F}
    \definecolor{ansi-white}{HTML}{C5C1B4}
    \definecolor{ansi-white-intense}{HTML}{A1A6B2}
    \definecolor{ansi-default-inverse-fg}{HTML}{FFFFFF}
    \definecolor{ansi-default-inverse-bg}{HTML}{000000}

    % commands and environments needed by pandoc snippets
    % extracted from the output of `pandoc -s`
    \providecommand{\tightlist}{%
      \setlength{\itemsep}{0pt}\setlength{\parskip}{0pt}}
    \DefineVerbatimEnvironment{Highlighting}{Verbatim}{commandchars=\\\{\}}
    % Add ',fontsize=\small' for more characters per line
    \newenvironment{Shaded}{}{}
    \newcommand{\KeywordTok}[1]{\textcolor[rgb]{0.00,0.44,0.13}{\textbf{{#1}}}}
    \newcommand{\DataTypeTok}[1]{\textcolor[rgb]{0.56,0.13,0.00}{{#1}}}
    \newcommand{\DecValTok}[1]{\textcolor[rgb]{0.25,0.63,0.44}{{#1}}}
    \newcommand{\BaseNTok}[1]{\textcolor[rgb]{0.25,0.63,0.44}{{#1}}}
    \newcommand{\FloatTok}[1]{\textcolor[rgb]{0.25,0.63,0.44}{{#1}}}
    \newcommand{\CharTok}[1]{\textcolor[rgb]{0.25,0.44,0.63}{{#1}}}
    \newcommand{\StringTok}[1]{\textcolor[rgb]{0.25,0.44,0.63}{{#1}}}
    \newcommand{\CommentTok}[1]{\textcolor[rgb]{0.38,0.63,0.69}{\textit{{#1}}}}
    \newcommand{\OtherTok}[1]{\textcolor[rgb]{0.00,0.44,0.13}{{#1}}}
    \newcommand{\AlertTok}[1]{\textcolor[rgb]{1.00,0.00,0.00}{\textbf{{#1}}}}
    \newcommand{\FunctionTok}[1]{\textcolor[rgb]{0.02,0.16,0.49}{{#1}}}
    \newcommand{\RegionMarkerTok}[1]{{#1}}
    \newcommand{\ErrorTok}[1]{\textcolor[rgb]{1.00,0.00,0.00}{\textbf{{#1}}}}
    \newcommand{\NormalTok}[1]{{#1}}
    
    % Additional commands for more recent versions of Pandoc
    \newcommand{\ConstantTok}[1]{\textcolor[rgb]{0.53,0.00,0.00}{{#1}}}
    \newcommand{\SpecialCharTok}[1]{\textcolor[rgb]{0.25,0.44,0.63}{{#1}}}
    \newcommand{\VerbatimStringTok}[1]{\textcolor[rgb]{0.25,0.44,0.63}{{#1}}}
    \newcommand{\SpecialStringTok}[1]{\textcolor[rgb]{0.73,0.40,0.53}{{#1}}}
    \newcommand{\ImportTok}[1]{{#1}}
    \newcommand{\DocumentationTok}[1]{\textcolor[rgb]{0.73,0.13,0.13}{\textit{{#1}}}}
    \newcommand{\AnnotationTok}[1]{\textcolor[rgb]{0.38,0.63,0.69}{\textbf{\textit{{#1}}}}}
    \newcommand{\CommentVarTok}[1]{\textcolor[rgb]{0.38,0.63,0.69}{\textbf{\textit{{#1}}}}}
    \newcommand{\VariableTok}[1]{\textcolor[rgb]{0.10,0.09,0.49}{{#1}}}
    \newcommand{\ControlFlowTok}[1]{\textcolor[rgb]{0.00,0.44,0.13}{\textbf{{#1}}}}
    \newcommand{\OperatorTok}[1]{\textcolor[rgb]{0.40,0.40,0.40}{{#1}}}
    \newcommand{\BuiltInTok}[1]{{#1}}
    \newcommand{\ExtensionTok}[1]{{#1}}
    \newcommand{\PreprocessorTok}[1]{\textcolor[rgb]{0.74,0.48,0.00}{{#1}}}
    \newcommand{\AttributeTok}[1]{\textcolor[rgb]{0.49,0.56,0.16}{{#1}}}
    \newcommand{\InformationTok}[1]{\textcolor[rgb]{0.38,0.63,0.69}{\textbf{\textit{{#1}}}}}
    \newcommand{\WarningTok}[1]{\textcolor[rgb]{0.38,0.63,0.69}{\textbf{\textit{{#1}}}}}
    
    
    % Define a nice break command that doesn't care if a line doesn't already
    % exist.
    \def\br{\hspace*{\fill} \\* }
    % Math Jax compatibility definitions
    \def\gt{>}
    \def\lt{<}
    \let\Oldtex\TeX
    \let\Oldlatex\LaTeX
    \renewcommand{\TeX}{\textrm{\Oldtex}}
    \renewcommand{\LaTeX}{\textrm{\Oldlatex}}
    % Document parameters
    % Document title
    \title{Untitled}
    
    
    
    
    
% Pygments definitions
\makeatletter
\def\PY@reset{\let\PY@it=\relax \let\PY@bf=\relax%
    \let\PY@ul=\relax \let\PY@tc=\relax%
    \let\PY@bc=\relax \let\PY@ff=\relax}
\def\PY@tok#1{\csname PY@tok@#1\endcsname}
\def\PY@toks#1+{\ifx\relax#1\empty\else%
    \PY@tok{#1}\expandafter\PY@toks\fi}
\def\PY@do#1{\PY@bc{\PY@tc{\PY@ul{%
    \PY@it{\PY@bf{\PY@ff{#1}}}}}}}
\def\PY#1#2{\PY@reset\PY@toks#1+\relax+\PY@do{#2}}

\expandafter\def\csname PY@tok@w\endcsname{\def\PY@tc##1{\textcolor[rgb]{0.73,0.73,0.73}{##1}}}
\expandafter\def\csname PY@tok@c\endcsname{\let\PY@it=\textit\def\PY@tc##1{\textcolor[rgb]{0.25,0.50,0.50}{##1}}}
\expandafter\def\csname PY@tok@cp\endcsname{\def\PY@tc##1{\textcolor[rgb]{0.74,0.48,0.00}{##1}}}
\expandafter\def\csname PY@tok@k\endcsname{\let\PY@bf=\textbf\def\PY@tc##1{\textcolor[rgb]{0.00,0.50,0.00}{##1}}}
\expandafter\def\csname PY@tok@kp\endcsname{\def\PY@tc##1{\textcolor[rgb]{0.00,0.50,0.00}{##1}}}
\expandafter\def\csname PY@tok@kt\endcsname{\def\PY@tc##1{\textcolor[rgb]{0.69,0.00,0.25}{##1}}}
\expandafter\def\csname PY@tok@o\endcsname{\def\PY@tc##1{\textcolor[rgb]{0.40,0.40,0.40}{##1}}}
\expandafter\def\csname PY@tok@ow\endcsname{\let\PY@bf=\textbf\def\PY@tc##1{\textcolor[rgb]{0.67,0.13,1.00}{##1}}}
\expandafter\def\csname PY@tok@nb\endcsname{\def\PY@tc##1{\textcolor[rgb]{0.00,0.50,0.00}{##1}}}
\expandafter\def\csname PY@tok@nf\endcsname{\def\PY@tc##1{\textcolor[rgb]{0.00,0.00,1.00}{##1}}}
\expandafter\def\csname PY@tok@nc\endcsname{\let\PY@bf=\textbf\def\PY@tc##1{\textcolor[rgb]{0.00,0.00,1.00}{##1}}}
\expandafter\def\csname PY@tok@nn\endcsname{\let\PY@bf=\textbf\def\PY@tc##1{\textcolor[rgb]{0.00,0.00,1.00}{##1}}}
\expandafter\def\csname PY@tok@ne\endcsname{\let\PY@bf=\textbf\def\PY@tc##1{\textcolor[rgb]{0.82,0.25,0.23}{##1}}}
\expandafter\def\csname PY@tok@nv\endcsname{\def\PY@tc##1{\textcolor[rgb]{0.10,0.09,0.49}{##1}}}
\expandafter\def\csname PY@tok@no\endcsname{\def\PY@tc##1{\textcolor[rgb]{0.53,0.00,0.00}{##1}}}
\expandafter\def\csname PY@tok@nl\endcsname{\def\PY@tc##1{\textcolor[rgb]{0.63,0.63,0.00}{##1}}}
\expandafter\def\csname PY@tok@ni\endcsname{\let\PY@bf=\textbf\def\PY@tc##1{\textcolor[rgb]{0.60,0.60,0.60}{##1}}}
\expandafter\def\csname PY@tok@na\endcsname{\def\PY@tc##1{\textcolor[rgb]{0.49,0.56,0.16}{##1}}}
\expandafter\def\csname PY@tok@nt\endcsname{\let\PY@bf=\textbf\def\PY@tc##1{\textcolor[rgb]{0.00,0.50,0.00}{##1}}}
\expandafter\def\csname PY@tok@nd\endcsname{\def\PY@tc##1{\textcolor[rgb]{0.67,0.13,1.00}{##1}}}
\expandafter\def\csname PY@tok@s\endcsname{\def\PY@tc##1{\textcolor[rgb]{0.73,0.13,0.13}{##1}}}
\expandafter\def\csname PY@tok@sd\endcsname{\let\PY@it=\textit\def\PY@tc##1{\textcolor[rgb]{0.73,0.13,0.13}{##1}}}
\expandafter\def\csname PY@tok@si\endcsname{\let\PY@bf=\textbf\def\PY@tc##1{\textcolor[rgb]{0.73,0.40,0.53}{##1}}}
\expandafter\def\csname PY@tok@se\endcsname{\let\PY@bf=\textbf\def\PY@tc##1{\textcolor[rgb]{0.73,0.40,0.13}{##1}}}
\expandafter\def\csname PY@tok@sr\endcsname{\def\PY@tc##1{\textcolor[rgb]{0.73,0.40,0.53}{##1}}}
\expandafter\def\csname PY@tok@ss\endcsname{\def\PY@tc##1{\textcolor[rgb]{0.10,0.09,0.49}{##1}}}
\expandafter\def\csname PY@tok@sx\endcsname{\def\PY@tc##1{\textcolor[rgb]{0.00,0.50,0.00}{##1}}}
\expandafter\def\csname PY@tok@m\endcsname{\def\PY@tc##1{\textcolor[rgb]{0.40,0.40,0.40}{##1}}}
\expandafter\def\csname PY@tok@gh\endcsname{\let\PY@bf=\textbf\def\PY@tc##1{\textcolor[rgb]{0.00,0.00,0.50}{##1}}}
\expandafter\def\csname PY@tok@gu\endcsname{\let\PY@bf=\textbf\def\PY@tc##1{\textcolor[rgb]{0.50,0.00,0.50}{##1}}}
\expandafter\def\csname PY@tok@gd\endcsname{\def\PY@tc##1{\textcolor[rgb]{0.63,0.00,0.00}{##1}}}
\expandafter\def\csname PY@tok@gi\endcsname{\def\PY@tc##1{\textcolor[rgb]{0.00,0.63,0.00}{##1}}}
\expandafter\def\csname PY@tok@gr\endcsname{\def\PY@tc##1{\textcolor[rgb]{1.00,0.00,0.00}{##1}}}
\expandafter\def\csname PY@tok@ge\endcsname{\let\PY@it=\textit}
\expandafter\def\csname PY@tok@gs\endcsname{\let\PY@bf=\textbf}
\expandafter\def\csname PY@tok@gp\endcsname{\let\PY@bf=\textbf\def\PY@tc##1{\textcolor[rgb]{0.00,0.00,0.50}{##1}}}
\expandafter\def\csname PY@tok@go\endcsname{\def\PY@tc##1{\textcolor[rgb]{0.53,0.53,0.53}{##1}}}
\expandafter\def\csname PY@tok@gt\endcsname{\def\PY@tc##1{\textcolor[rgb]{0.00,0.27,0.87}{##1}}}
\expandafter\def\csname PY@tok@err\endcsname{\def\PY@bc##1{\setlength{\fboxsep}{0pt}\fcolorbox[rgb]{1.00,0.00,0.00}{1,1,1}{\strut ##1}}}
\expandafter\def\csname PY@tok@kc\endcsname{\let\PY@bf=\textbf\def\PY@tc##1{\textcolor[rgb]{0.00,0.50,0.00}{##1}}}
\expandafter\def\csname PY@tok@kd\endcsname{\let\PY@bf=\textbf\def\PY@tc##1{\textcolor[rgb]{0.00,0.50,0.00}{##1}}}
\expandafter\def\csname PY@tok@kn\endcsname{\let\PY@bf=\textbf\def\PY@tc##1{\textcolor[rgb]{0.00,0.50,0.00}{##1}}}
\expandafter\def\csname PY@tok@kr\endcsname{\let\PY@bf=\textbf\def\PY@tc##1{\textcolor[rgb]{0.00,0.50,0.00}{##1}}}
\expandafter\def\csname PY@tok@bp\endcsname{\def\PY@tc##1{\textcolor[rgb]{0.00,0.50,0.00}{##1}}}
\expandafter\def\csname PY@tok@fm\endcsname{\def\PY@tc##1{\textcolor[rgb]{0.00,0.00,1.00}{##1}}}
\expandafter\def\csname PY@tok@vc\endcsname{\def\PY@tc##1{\textcolor[rgb]{0.10,0.09,0.49}{##1}}}
\expandafter\def\csname PY@tok@vg\endcsname{\def\PY@tc##1{\textcolor[rgb]{0.10,0.09,0.49}{##1}}}
\expandafter\def\csname PY@tok@vi\endcsname{\def\PY@tc##1{\textcolor[rgb]{0.10,0.09,0.49}{##1}}}
\expandafter\def\csname PY@tok@vm\endcsname{\def\PY@tc##1{\textcolor[rgb]{0.10,0.09,0.49}{##1}}}
\expandafter\def\csname PY@tok@sa\endcsname{\def\PY@tc##1{\textcolor[rgb]{0.73,0.13,0.13}{##1}}}
\expandafter\def\csname PY@tok@sb\endcsname{\def\PY@tc##1{\textcolor[rgb]{0.73,0.13,0.13}{##1}}}
\expandafter\def\csname PY@tok@sc\endcsname{\def\PY@tc##1{\textcolor[rgb]{0.73,0.13,0.13}{##1}}}
\expandafter\def\csname PY@tok@dl\endcsname{\def\PY@tc##1{\textcolor[rgb]{0.73,0.13,0.13}{##1}}}
\expandafter\def\csname PY@tok@s2\endcsname{\def\PY@tc##1{\textcolor[rgb]{0.73,0.13,0.13}{##1}}}
\expandafter\def\csname PY@tok@sh\endcsname{\def\PY@tc##1{\textcolor[rgb]{0.73,0.13,0.13}{##1}}}
\expandafter\def\csname PY@tok@s1\endcsname{\def\PY@tc##1{\textcolor[rgb]{0.73,0.13,0.13}{##1}}}
\expandafter\def\csname PY@tok@mb\endcsname{\def\PY@tc##1{\textcolor[rgb]{0.40,0.40,0.40}{##1}}}
\expandafter\def\csname PY@tok@mf\endcsname{\def\PY@tc##1{\textcolor[rgb]{0.40,0.40,0.40}{##1}}}
\expandafter\def\csname PY@tok@mh\endcsname{\def\PY@tc##1{\textcolor[rgb]{0.40,0.40,0.40}{##1}}}
\expandafter\def\csname PY@tok@mi\endcsname{\def\PY@tc##1{\textcolor[rgb]{0.40,0.40,0.40}{##1}}}
\expandafter\def\csname PY@tok@il\endcsname{\def\PY@tc##1{\textcolor[rgb]{0.40,0.40,0.40}{##1}}}
\expandafter\def\csname PY@tok@mo\endcsname{\def\PY@tc##1{\textcolor[rgb]{0.40,0.40,0.40}{##1}}}
\expandafter\def\csname PY@tok@ch\endcsname{\let\PY@it=\textit\def\PY@tc##1{\textcolor[rgb]{0.25,0.50,0.50}{##1}}}
\expandafter\def\csname PY@tok@cm\endcsname{\let\PY@it=\textit\def\PY@tc##1{\textcolor[rgb]{0.25,0.50,0.50}{##1}}}
\expandafter\def\csname PY@tok@cpf\endcsname{\let\PY@it=\textit\def\PY@tc##1{\textcolor[rgb]{0.25,0.50,0.50}{##1}}}
\expandafter\def\csname PY@tok@c1\endcsname{\let\PY@it=\textit\def\PY@tc##1{\textcolor[rgb]{0.25,0.50,0.50}{##1}}}
\expandafter\def\csname PY@tok@cs\endcsname{\let\PY@it=\textit\def\PY@tc##1{\textcolor[rgb]{0.25,0.50,0.50}{##1}}}

\def\PYZbs{\char`\\}
\def\PYZus{\char`\_}
\def\PYZob{\char`\{}
\def\PYZcb{\char`\}}
\def\PYZca{\char`\^}
\def\PYZam{\char`\&}
\def\PYZlt{\char`\<}
\def\PYZgt{\char`\>}
\def\PYZsh{\char`\#}
\def\PYZpc{\char`\%}
\def\PYZdl{\char`\$}
\def\PYZhy{\char`\-}
\def\PYZsq{\char`\'}
\def\PYZdq{\char`\"}
\def\PYZti{\char`\~}
% for compatibility with earlier versions
\def\PYZat{@}
\def\PYZlb{[}
\def\PYZrb{]}
\makeatother


    % For linebreaks inside Verbatim environment from package fancyvrb. 
    \makeatletter
        \newbox\Wrappedcontinuationbox 
        \newbox\Wrappedvisiblespacebox 
        \newcommand*\Wrappedvisiblespace {\textcolor{red}{\textvisiblespace}} 
        \newcommand*\Wrappedcontinuationsymbol {\textcolor{red}{\llap{\tiny$\m@th\hookrightarrow$}}} 
        \newcommand*\Wrappedcontinuationindent {3ex } 
        \newcommand*\Wrappedafterbreak {\kern\Wrappedcontinuationindent\copy\Wrappedcontinuationbox} 
        % Take advantage of the already applied Pygments mark-up to insert 
        % potential linebreaks for TeX processing. 
        %        {, <, #, %, $, ' and ": go to next line. 
        %        _, }, ^, &, >, - and ~: stay at end of broken line. 
        % Use of \textquotesingle for straight quote. 
        \newcommand*\Wrappedbreaksatspecials {% 
            \def\PYGZus{\discretionary{\char`\_}{\Wrappedafterbreak}{\char`\_}}% 
            \def\PYGZob{\discretionary{}{\Wrappedafterbreak\char`\{}{\char`\{}}% 
            \def\PYGZcb{\discretionary{\char`\}}{\Wrappedafterbreak}{\char`\}}}% 
            \def\PYGZca{\discretionary{\char`\^}{\Wrappedafterbreak}{\char`\^}}% 
            \def\PYGZam{\discretionary{\char`\&}{\Wrappedafterbreak}{\char`\&}}% 
            \def\PYGZlt{\discretionary{}{\Wrappedafterbreak\char`\<}{\char`\<}}% 
            \def\PYGZgt{\discretionary{\char`\>}{\Wrappedafterbreak}{\char`\>}}% 
            \def\PYGZsh{\discretionary{}{\Wrappedafterbreak\char`\#}{\char`\#}}% 
            \def\PYGZpc{\discretionary{}{\Wrappedafterbreak\char`\%}{\char`\%}}% 
            \def\PYGZdl{\discretionary{}{\Wrappedafterbreak\char`\$}{\char`\$}}% 
            \def\PYGZhy{\discretionary{\char`\-}{\Wrappedafterbreak}{\char`\-}}% 
            \def\PYGZsq{\discretionary{}{\Wrappedafterbreak\textquotesingle}{\textquotesingle}}% 
            \def\PYGZdq{\discretionary{}{\Wrappedafterbreak\char`\"}{\char`\"}}% 
            \def\PYGZti{\discretionary{\char`\~}{\Wrappedafterbreak}{\char`\~}}% 
        } 
        % Some characters . , ; ? ! / are not pygmentized. 
        % This macro makes them "active" and they will insert potential linebreaks 
        \newcommand*\Wrappedbreaksatpunct {% 
            \lccode`\~`\.\lowercase{\def~}{\discretionary{\hbox{\char`\.}}{\Wrappedafterbreak}{\hbox{\char`\.}}}% 
            \lccode`\~`\,\lowercase{\def~}{\discretionary{\hbox{\char`\,}}{\Wrappedafterbreak}{\hbox{\char`\,}}}% 
            \lccode`\~`\;\lowercase{\def~}{\discretionary{\hbox{\char`\;}}{\Wrappedafterbreak}{\hbox{\char`\;}}}% 
            \lccode`\~`\:\lowercase{\def~}{\discretionary{\hbox{\char`\:}}{\Wrappedafterbreak}{\hbox{\char`\:}}}% 
            \lccode`\~`\?\lowercase{\def~}{\discretionary{\hbox{\char`\?}}{\Wrappedafterbreak}{\hbox{\char`\?}}}% 
            \lccode`\~`\!\lowercase{\def~}{\discretionary{\hbox{\char`\!}}{\Wrappedafterbreak}{\hbox{\char`\!}}}% 
            \lccode`\~`\/\lowercase{\def~}{\discretionary{\hbox{\char`\/}}{\Wrappedafterbreak}{\hbox{\char`\/}}}% 
            \catcode`\.\active
            \catcode`\,\active 
            \catcode`\;\active
            \catcode`\:\active
            \catcode`\?\active
            \catcode`\!\active
            \catcode`\/\active 
            \lccode`\~`\~ 	
        }
    \makeatother

    \let\OriginalVerbatim=\Verbatim
    \makeatletter
    \renewcommand{\Verbatim}[1][1]{%
        %\parskip\z@skip
        \sbox\Wrappedcontinuationbox {\Wrappedcontinuationsymbol}%
        \sbox\Wrappedvisiblespacebox {\FV@SetupFont\Wrappedvisiblespace}%
        \def\FancyVerbFormatLine ##1{\hsize\linewidth
            \vtop{\raggedright\hyphenpenalty\z@\exhyphenpenalty\z@
                \doublehyphendemerits\z@\finalhyphendemerits\z@
                \strut ##1\strut}%
        }%
        % If the linebreak is at a space, the latter will be displayed as visible
        % space at end of first line, and a continuation symbol starts next line.
        % Stretch/shrink are however usually zero for typewriter font.
        \def\FV@Space {%
            \nobreak\hskip\z@ plus\fontdimen3\font minus\fontdimen4\font
            \discretionary{\copy\Wrappedvisiblespacebox}{\Wrappedafterbreak}
            {\kern\fontdimen2\font}%
        }%
        
        % Allow breaks at special characters using \PYG... macros.
        \Wrappedbreaksatspecials
        % Breaks at punctuation characters . , ; ? ! and / need catcode=\active 	
        \OriginalVerbatim[#1,codes*=\Wrappedbreaksatpunct]%
    }
    \makeatother

    % Exact colors from NB
    \definecolor{incolor}{HTML}{303F9F}
    \definecolor{outcolor}{HTML}{D84315}
    \definecolor{cellborder}{HTML}{CFCFCF}
    \definecolor{cellbackground}{HTML}{F7F7F7}
    
    % prompt
    \makeatletter
    \newcommand{\boxspacing}{\kern\kvtcb@left@rule\kern\kvtcb@boxsep}
    \makeatother
    \newcommand{\prompt}[4]{
        \ttfamily\llap{{\color{#2}[#3]:\hspace{3pt}#4}}\vspace{-\baselineskip}
    }
    

    
    % Prevent overflowing lines due to hard-to-break entities
    \sloppy 
    % Setup hyperref package
    \hypersetup{
      breaklinks=true,  % so long urls are correctly broken across lines
      colorlinks=true,
      urlcolor=urlcolor,
      linkcolor=linkcolor,
      citecolor=citecolor,
      }
    % Slightly bigger margins than the latex defaults
    
    \geometry{verbose,tmargin=1in,bmargin=1in,lmargin=1in,rmargin=1in}
    
    

\begin{document}
    
    \maketitle
    
    

    
    \hypertarget{particula-en-una-caja-unidimensional}{%
\section{Particula en una caja
unidimensional}\label{particula-en-una-caja-unidimensional}}

    Este problema lo podemos separar en tres regiones distintas, la solucion
para la region I y III es la misma.

la ecuacion de schodinger esta dada por

\[\left(-\frac{\hbar ^2}{2m}\frac{\partial ^2}{\partial x^2}+V\right)\Psi (x)=E\Psi (x)\]

para nuestro caso la solucion a la region I y III contiente un potencial
\(V_0\) constante, de esta manera la ecuacion de schrodinger toma la
forma;

\[\left(-\frac{\hbar ^2}{2m}\frac{\partial ^2}{\partial x^2}+V_0\right)\Psi (x)=E\Psi (x)\]

y despejando la segunda derivada que deja de ser parcial y pasa a ser
una total pues solo estamos trabajando en una coordenada.

\[\frac{d^2\Psi (x)}{dx^2}=\frac{-2m}{\hbar ^2}(E-V_0)\Psi (x)\]

para dicha ecuacion diferencial la solucion es de la forma \(e^{kx}\)
donde \(k=\sqrt{\frac{2m(V_0-E)}{\hbar ^2}}\), de esta manera las
soluciones para la region I y II es:

\[\Psi _I(x)=A_Ie^{kx}+B_Ie^{-kx}\]

\[\Psi _{III}(x)=A_{III}e^{kx}+B_{III}e^{-kx}\]

el valor de \(k\) siempre es positivo, pues de la grafica nos podemos
dar cuenta que el valor de la energia siempre es menor al del potencial,
\(E<V_0\)

para la region II tenemos que la ecuacion diferencial es de la forma:

\[\frac{d^2\Psi}{dx^2}=\frac{-2mE}{\hbar ^2}\Psi (x)\]

la cual tiene como sulucion una exponencial imaginaria, \(e^{ik^´x}\),
donde \(k^´=\sqrt{\frac{2mE}{\hbar ^2}}\)

entonces la solucion para la region \(II\) region es:

\[\psi _{II}(x)=A_{II}sin(k'x)+B_{II}cos(k'x)\]

\hypertarget{solucion-para-cada-region}{%
\subsubsection{Solucion para cada
region}\label{solucion-para-cada-region}}

como condiciones de frontera tenemos que las funciones se tienen que
hacer cero en el infinito, por tanto para las regiones \(I\) y \(II\)
algunas constantes se deben hacer cero, entoncess \(B_I=A_{III}=0\). De
esta manera las soluciones momentaneas para cada region son:

\[ \Psi _I=A_Ie^{kx}\]

\[ \Psi _{II}=A_{II}sin(k'x)+B_{II}cos(k'x)\]

\[ \Psi _{III}=B_{III}e^{-kx}\]

\hypertarget{aplicando-las-condiciones-de-frontera}{%
\subsubsection{Aplicando las condiciones de
frontera}\label{aplicando-las-condiciones-de-frontera}}

la funcion y la derivada debe ser continua en \(x=-a/2\) y \(x=a/2\),
entonces:

\[\Psi _I(-a/2)=\Psi _{II}(-a/2)\]

\[\Psi _{II}(a/2)=\Psi _{III}(a/2)\]

\[\Psi _{I}^{'}(-a/2)=\Psi _{II}^{'}(-a/2)\]

\[\Psi _{II}^{'}(a/2)=\Psi _{III}^{'}(a/2)\]

entonces:

\[A_Ie^{-a/2 k}=B_{II}cos(a/2 k')-A_{II}sin(a/2 k')\]

\[B_{III}e^{-a/2 k}=B_{II}cos(a/2 k')+A_{II}sin(a/2 k')\]

\[kA_I e^{-a/2 k}=k'(A_{II}cos(a/2 k')+B_{II}sin(a/2 k'))\]

juntando la primer y tercer ecuacin tenemos que:

\[k'[A_{II}cos(a/2 k')+B_{II}sin(a/2 k')]=k[B_{II}cos(a/2 k')-A_{II}sin(a/2 k')]\]

y de la segunda y cuarta ecuacion obtenemos

\[k'[A_{II}cos(a/2 k')-B_{II}sin(a/2 k')]=k[-B_{II}cos(a/2 k')-A_{II}sin(a/2 k')]\]

para poder que el par de ecuaciones anteriores se cumplan, tenemos que
asegurar que el producto entre \(A_{II}\) y \(B_{II}\) sea cero,
\(A_{II}B_{II}=0\), lo que nos arroja dos tipos de soluciones diferentes
para \(\Psi_{II}\), una con seno y otra con coseno.

cuando \(A_{II}=0\) de las ecuaciones anteriores obtenemos que:

\[k'sin(a/2 k')=k cos(a/2 k')\]

dicha ecucion lleva a una trascendental, de la cual se pueden obtener
los autovalores de energia que estan dados por la ecuacion:

\[E=\frac{\hbar ^2 \pi ^2}{2ma^2}(2n+1)^2\]

utilizando unidades atomicas \(\hbar=1\), \(m=1\)

\[E=\frac{\pi ^2}{2a^2}(2n+1)^2\]

\hypertarget{normalizacion}{%
\subsubsection{Normalizacion}\label{normalizacion}}

las funciones con las que trabajamos deben estar normalizadas, esto es
que:

\[\int{|\psi (x)|^2dx}=1\]

de aqui se puede obtener el valor de las constantes \(A_{I}\),
\(A_{II}\), \(B_{II}\) y \(B_{III}\). las condiciones que deben cumplir
son:

\[A_I=\sqrt{ke^{-k a/2}}\]

\[B_{III}=\sqrt{ke^{-ka/2}}\]

\[\frac{A_{II}^2}{2}\left(a-\frac{sin(ak')}{k'}\right)+\frac{B_{II}^2}{2}\left(a+\frac{sin(ak')}{k'}\right)=1\]

haciendo \(A_{II}=0\) tenemos que \(B_{II}\) es:

\[B_{II}=\sqrt\frac{2}{a+\frac{sin(ak')}{k'}}\]

se tomo \(A_{II}=0\) ya que la caja se encuentra entre \(-a/2\) y
\(a/2\) por tanto la solucion debe ser par y de las soluciones
propuestas para la region \(II\) la unica que cumple es coseno, asi que
anulamos el seno.

\hypertarget{solucion-para-la-caja}{%
\subsection{solucion para la caja}\label{solucion-para-la-caja}}

con todo el desarrollo realizado anteriormente las soluciones fuera de
la caja son:

\[ \Psi _I=\sqrt{ke^{-k a/2}}e^{kx}\]

\[ \Psi _{III}=\sqrt{ke^{-ka/2}}e^{-kx}\]

\[ \Psi_{II}=\sqrt\frac{2}{a+\frac{sin(ak')}{k'}} cos(k'x)\]

donde

\[k=\sqrt{\frac{2m(V_0-E)}{\hbar ^2}}\]

\[k'=\sqrt{\frac{2mE}{\hbar ^2}}\]

    \begin{tcolorbox}[breakable, size=fbox, boxrule=1pt, pad at break*=1mm,colback=cellbackground, colframe=cellborder]
\prompt{In}{incolor}{346}{\boxspacing}
\begin{Verbatim}[commandchars=\\\{\}]
\PY{k+kn}{import} \PY{n+nn}{numpy} \PY{k}{as} \PY{n+nn}{np}
\PY{k+kn}{import} \PY{n+nn}{matplotlib}\PY{n+nn}{.}\PY{n+nn}{pyplot} \PY{k}{as} \PY{n+nn}{plt}
\PY{k+kn}{from} \PY{n+nn}{scipy}\PY{n+nn}{.}\PY{n+nn}{integrate} \PY{k+kn}{import} \PY{n}{solve\PYZus{}bvp}
\end{Verbatim}
\end{tcolorbox}

    \begin{tcolorbox}[breakable, size=fbox, boxrule=1pt, pad at break*=1mm,colback=cellbackground, colframe=cellborder]
\prompt{In}{incolor}{357}{\boxspacing}
\begin{Verbatim}[commandchars=\\\{\}]
\PY{n}{a}\PY{o}{=}\PY{l+m+mi}{2} \PY{c+c1}{\PYZsh{}ancho de la caja}

\PY{c+c1}{\PYZsh{}\PYZsh{} potencial de la caja }
\PY{k}{def} \PY{n+nf}{potencial}\PY{p}{(}\PY{p}{)}\PY{p}{:}
    \PY{n}{V\PYZus{}0}\PY{o}{=}\PY{l+m+mi}{300} \PY{c+c1}{\PYZsh{} valor del potencial fuera de la caja }
    \PY{k}{return} \PY{n}{V\PYZus{}0}


\PY{c+c1}{\PYZsh{}\PYZsh{} cosntante k para las regiones I y III}
\PY{k}{def} \PY{n+nf}{k}\PY{p}{(}\PY{n}{n}\PY{p}{)}\PY{p}{:}
    \PY{n}{E}\PY{o}{=}\PY{n}{np}\PY{o}{.}\PY{n}{pi}\PY{o}{*}\PY{o}{*}\PY{l+m+mi}{2}\PY{o}{*}\PY{p}{(}\PY{l+m+mi}{2}\PY{o}{*}\PY{n}{n}\PY{o}{+}\PY{l+m+mi}{1}\PY{p}{)}\PY{o}{*}\PY{o}{*}\PY{l+m+mi}{2}\PY{o}{/}\PY{p}{(}\PY{l+m+mi}{2}\PY{o}{*}\PY{n}{a}\PY{o}{*}\PY{o}{*}\PY{l+m+mi}{2}\PY{p}{)}\PY{c+c1}{\PYZsh{}valor de autoenergia}
    \PY{n}{k}\PY{o}{=}\PY{n}{np}\PY{o}{.}\PY{n}{sqrt}\PY{p}{(}\PY{p}{(}\PY{l+m+mi}{2}\PY{o}{*}\PY{p}{(}\PY{n}{potencial}\PY{p}{(}\PY{p}{)}\PY{o}{\PYZhy{}}\PY{n}{E}\PY{p}{)}\PY{p}{)}\PY{p}{)} \PY{c+c1}{\PYZsh{}k en forma a\PYZhy{}dimensional}

    \PY{k}{return} \PY{n}{k}

\PY{c+c1}{\PYZsh{}cosntante k\PYZsq{} para la region II}
\PY{k}{def} \PY{n+nf}{kp}\PY{p}{(}\PY{n}{n}\PY{p}{)}\PY{p}{:}
    \PY{n}{E}\PY{o}{=}\PY{p}{(}\PY{n}{np}\PY{o}{.}\PY{n}{pi}\PY{o}{*}\PY{o}{*}\PY{l+m+mi}{2}\PY{o}{*}\PY{p}{(}\PY{l+m+mi}{2}\PY{o}{*}\PY{n}{n}\PY{o}{+}\PY{l+m+mi}{1}\PY{p}{)}\PY{o}{*}\PY{o}{*}\PY{l+m+mi}{2}\PY{p}{)}\PY{o}{/}\PY{p}{(}\PY{l+m+mi}{2}\PY{o}{*}\PY{n}{a}\PY{o}{*}\PY{o}{*}\PY{l+m+mi}{2}\PY{p}{)} \PY{c+c1}{\PYZsh{}valor de autoenergia}
    \PY{n}{kp}\PY{o}{=}\PY{n}{np}\PY{o}{.}\PY{n}{sqrt}\PY{p}{(}\PY{p}{(}\PY{l+m+mi}{2}\PY{o}{*}\PY{n}{E}\PY{p}{)}\PY{p}{)} \PY{c+c1}{\PYZsh{}k\PYZsq{} en forma a\PYZhy{}dimensional}
    \PY{k}{return} \PY{n}{kp}


\PY{c+c1}{\PYZsh{} funciones que salieron como solucion a las regiones I y III}

\PY{k}{def} \PY{n+nf}{psi1}\PY{p}{(}\PY{n}{x}\PY{p}{,}\PY{n}{n}\PY{p}{)}\PY{p}{:}
    \PY{n}{A\PYZus{}1}\PY{o}{=}\PY{n}{np}\PY{o}{.}\PY{n}{sqrt}\PY{p}{(}\PY{n}{k}\PY{p}{(}\PY{n}{n}\PY{p}{)}\PY{o}{*}\PY{n}{np}\PY{o}{.}\PY{n}{e}\PY{o}{*}\PY{o}{*}\PY{p}{(}\PY{o}{\PYZhy{}}\PY{n}{k}\PY{p}{(}\PY{n}{n}\PY{p}{)}\PY{o}{*}\PY{n}{a}\PY{o}{/}\PY{l+m+mi}{2}\PY{p}{)}\PY{p}{)}
    \PY{n}{psi1}\PY{o}{=}\PY{n}{A\PYZus{}1}\PY{o}{*}\PY{n}{np}\PY{o}{.}\PY{n}{e}\PY{o}{*}\PY{o}{*}\PY{p}{(}\PY{n}{k}\PY{p}{(}\PY{n}{n}\PY{p}{)}\PY{o}{*}\PY{n}{x}\PY{p}{)}
    \PY{k}{return}\PY{p}{(}\PY{n}{psi1}\PY{p}{)}

\PY{k}{def} \PY{n+nf}{psi3}\PY{p}{(}\PY{n}{x}\PY{p}{,}\PY{n}{n}\PY{p}{)}\PY{p}{:}
    \PY{n}{B\PYZus{}3}\PY{o}{=}\PY{n}{np}\PY{o}{.}\PY{n}{sqrt}\PY{p}{(}\PY{n}{k}\PY{p}{(}\PY{n}{n}\PY{p}{)}\PY{o}{*}\PY{n}{np}\PY{o}{.}\PY{n}{e}\PY{o}{*}\PY{o}{*}\PY{p}{(}\PY{o}{\PYZhy{}}\PY{n}{k}\PY{p}{(}\PY{n}{n}\PY{p}{)}\PY{o}{*}\PY{n}{a}\PY{o}{/}\PY{l+m+mi}{2}\PY{p}{)}\PY{p}{)}
    \PY{n}{psi3}\PY{o}{=}\PY{n}{B\PYZus{}3}\PY{o}{*}\PY{n}{np}\PY{o}{.}\PY{n}{e}\PY{o}{*}\PY{o}{*}\PY{p}{(}\PY{o}{\PYZhy{}}\PY{n}{k}\PY{p}{(}\PY{n}{n}\PY{p}{)}\PY{o}{*}\PY{n}{x}\PY{p}{)}
    \PY{k}{return}\PY{p}{(}\PY{n}{psi3}\PY{p}{)}

\PY{c+c1}{\PYZsh{}funcion que salio como solucion a la region II }

\PY{k}{def} \PY{n+nf}{psi2}\PY{p}{(}\PY{n}{x}\PY{p}{,}\PY{n}{n}\PY{p}{)}\PY{p}{:}
    \PY{n}{A\PYZus{}2}\PY{o}{=}\PY{l+m+mi}{0}
    \PY{n}{B\PYZus{}2}\PY{o}{=}\PY{n}{np}\PY{o}{.}\PY{n}{sqrt}\PY{p}{(}\PY{l+m+mi}{2}\PY{o}{/}\PY{p}{(}\PY{n}{a}\PY{o}{+}\PY{p}{(}\PY{n}{np}\PY{o}{.}\PY{n}{sin}\PY{p}{(}\PY{n}{kp}\PY{p}{(}\PY{n}{n}\PY{p}{)}\PY{p}{)}\PY{p}{)}\PY{o}{/}\PY{p}{(}\PY{n}{kp}\PY{p}{(}\PY{n}{n}\PY{p}{)}\PY{p}{)}\PY{p}{)}\PY{p}{)}
    \PY{n}{psi2}\PY{o}{=}\PY{n}{A\PYZus{}2}\PY{o}{*}\PY{n}{np}\PY{o}{.}\PY{n}{sin}\PY{p}{(}\PY{n}{kp}\PY{p}{(}\PY{n}{n}\PY{p}{)}\PY{o}{*}\PY{n}{x}\PY{p}{)}\PY{o}{+}\PY{n}{B\PYZus{}2}\PY{o}{*}\PY{n}{np}\PY{o}{.}\PY{n}{cos}\PY{p}{(}\PY{n}{kp}\PY{p}{(}\PY{n}{n}\PY{p}{)}\PY{o}{*}\PY{n}{x}\PY{p}{)}
    \PY{k}{return}\PY{p}{(}\PY{n}{psi2}\PY{p}{)}
   



\PY{c+c1}{\PYZsh{} rangos para graficar }

\PY{n}{x1}\PY{o}{=}\PY{n}{np}\PY{o}{.}\PY{n}{arange}\PY{p}{(}\PY{o}{\PYZhy{}}\PY{n}{a}\PY{p}{,}\PY{o}{\PYZhy{}}\PY{n}{a}\PY{o}{/}\PY{l+m+mi}{2}\PY{p}{,}\PY{l+m+mf}{0.0001}\PY{p}{)}
\PY{n}{x3}\PY{o}{=}\PY{n}{np}\PY{o}{.}\PY{n}{arange}\PY{p}{(}\PY{n}{a}\PY{o}{/}\PY{l+m+mi}{2}\PY{p}{,}\PY{n}{a}\PY{p}{,}\PY{l+m+mf}{0.0001}\PY{p}{)}
\PY{n}{x2}\PY{o}{=}\PY{n}{np}\PY{o}{.}\PY{n}{arange}\PY{p}{(}\PY{o}{\PYZhy{}}\PY{n}{a}\PY{o}{/}\PY{l+m+mi}{2}\PY{p}{,}\PY{n}{a}\PY{o}{/}\PY{l+m+mi}{2}\PY{p}{,}\PY{l+m+mf}{0.0001}\PY{p}{)}
\end{Verbatim}
\end{tcolorbox}

    \hypertarget{para-n0}{%
\subsection{Para n=0}\label{para-n0}}

    \begin{tcolorbox}[breakable, size=fbox, boxrule=1pt, pad at break*=1mm,colback=cellbackground, colframe=cellborder]
\prompt{In}{incolor}{372}{\boxspacing}
\begin{Verbatim}[commandchars=\\\{\}]
\PY{n}{fig}\PY{o}{=}\PY{n}{plt}\PY{o}{.}\PY{n}{subplots}\PY{p}{(}\PY{p}{)}
\PY{n}{plt}\PY{o}{.}\PY{n}{plot}\PY{p}{(}\PY{n}{x1}\PY{p}{,}\PY{n}{psi1}\PY{p}{(}\PY{n}{x1}\PY{p}{,}\PY{l+m+mi}{0}\PY{p}{)}\PY{p}{)}
\PY{n}{plt}\PY{o}{.}\PY{n}{plot}\PY{p}{(}\PY{n}{x3}\PY{p}{,}\PY{n}{psi3}\PY{p}{(}\PY{n}{x3}\PY{p}{,}\PY{l+m+mi}{0}\PY{p}{)}\PY{p}{)}
\PY{n}{plt}\PY{o}{.}\PY{n}{plot}\PY{p}{(}\PY{n}{x2}\PY{p}{,}\PY{n}{psi2}\PY{p}{(}\PY{n}{x2}\PY{p}{,}\PY{l+m+mi}{0}\PY{p}{)}\PY{p}{)}
\end{Verbatim}
\end{tcolorbox}

            \begin{tcolorbox}[breakable, size=fbox, boxrule=.5pt, pad at break*=1mm, opacityfill=0]
\prompt{Out}{outcolor}{372}{\boxspacing}
\begin{Verbatim}[commandchars=\\\{\}]
[<matplotlib.lines.Line2D at 0x1855fd90>]
\end{Verbatim}
\end{tcolorbox}
        
    \begin{center}
    \adjustimage{max size={0.9\linewidth}{0.9\paperheight}}{output_5_1.png}
    \end{center}
    { \hspace*{\fill} \\}
    
    \hypertarget{para-n1}{%
\subsection{Para n=1}\label{para-n1}}

    \begin{tcolorbox}[breakable, size=fbox, boxrule=1pt, pad at break*=1mm,colback=cellbackground, colframe=cellborder]
\prompt{In}{incolor}{359}{\boxspacing}
\begin{Verbatim}[commandchars=\\\{\}]
\PY{n}{fig}\PY{o}{=}\PY{n}{plt}\PY{o}{.}\PY{n}{figure}\PY{p}{(}\PY{p}{)}
\PY{n}{plt}\PY{o}{.}\PY{n}{plot}\PY{p}{(}\PY{n}{x1}\PY{p}{,}\PY{n}{psi1}\PY{p}{(}\PY{n}{x1}\PY{p}{,}\PY{l+m+mi}{1}\PY{p}{)}\PY{p}{)}
\PY{n}{plt}\PY{o}{.}\PY{n}{plot}\PY{p}{(}\PY{n}{x3}\PY{p}{,}\PY{n}{psi3}\PY{p}{(}\PY{n}{x3}\PY{p}{,}\PY{l+m+mi}{1}\PY{p}{)}\PY{p}{)}
\PY{n}{plt}\PY{o}{.}\PY{n}{plot}\PY{p}{(}\PY{n}{x2}\PY{p}{,}\PY{n}{psi2}\PY{p}{(}\PY{n}{x2}\PY{p}{,}\PY{l+m+mi}{1}\PY{p}{)}\PY{p}{)}
\end{Verbatim}
\end{tcolorbox}

            \begin{tcolorbox}[breakable, size=fbox, boxrule=.5pt, pad at break*=1mm, opacityfill=0]
\prompt{Out}{outcolor}{359}{\boxspacing}
\begin{Verbatim}[commandchars=\\\{\}]
[<matplotlib.lines.Line2D at 0x1704d148>]
\end{Verbatim}
\end{tcolorbox}
        
    \begin{center}
    \adjustimage{max size={0.9\linewidth}{0.9\paperheight}}{output_7_1.png}
    \end{center}
    { \hspace*{\fill} \\}
    
    \hypertarget{para-n2}{%
\subsection{Para n=2}\label{para-n2}}

    \begin{tcolorbox}[breakable, size=fbox, boxrule=1pt, pad at break*=1mm,colback=cellbackground, colframe=cellborder]
\prompt{In}{incolor}{360}{\boxspacing}
\begin{Verbatim}[commandchars=\\\{\}]
\PY{n}{fig}\PY{o}{=}\PY{n}{plt}\PY{o}{.}\PY{n}{figure}\PY{p}{(}\PY{p}{)}
\PY{n}{plt}\PY{o}{.}\PY{n}{plot}\PY{p}{(}\PY{n}{x1}\PY{p}{,}\PY{n}{psi1}\PY{p}{(}\PY{n}{x1}\PY{p}{,}\PY{l+m+mi}{2}\PY{p}{)}\PY{p}{)}
\PY{n}{plt}\PY{o}{.}\PY{n}{plot}\PY{p}{(}\PY{n}{x3}\PY{p}{,}\PY{n}{psi3}\PY{p}{(}\PY{n}{x3}\PY{p}{,}\PY{l+m+mi}{2}\PY{p}{)}\PY{p}{)}
\PY{n}{plt}\PY{o}{.}\PY{n}{plot}\PY{p}{(}\PY{n}{x2}\PY{p}{,}\PY{n}{psi2}\PY{p}{(}\PY{n}{x2}\PY{p}{,}\PY{l+m+mi}{2}\PY{p}{)}\PY{p}{)}
\end{Verbatim}
\end{tcolorbox}

            \begin{tcolorbox}[breakable, size=fbox, boxrule=.5pt, pad at break*=1mm, opacityfill=0]
\prompt{Out}{outcolor}{360}{\boxspacing}
\begin{Verbatim}[commandchars=\\\{\}]
[<matplotlib.lines.Line2D at 0xb06c70>]
\end{Verbatim}
\end{tcolorbox}
        
    \begin{center}
    \adjustimage{max size={0.9\linewidth}{0.9\paperheight}}{output_9_1.png}
    \end{center}
    { \hspace*{\fill} \\}
    
    \hypertarget{para-n3}{%
\subsection{Para n=3}\label{para-n3}}

    \begin{tcolorbox}[breakable, size=fbox, boxrule=1pt, pad at break*=1mm,colback=cellbackground, colframe=cellborder]
\prompt{In}{incolor}{411}{\boxspacing}
\begin{Verbatim}[commandchars=\\\{\}]
\PY{n}{fig}\PY{o}{=}\PY{n}{plt}\PY{o}{.}\PY{n}{figure}\PY{p}{(}\PY{p}{)}
\PY{n}{plt}\PY{o}{.}\PY{n}{plot}\PY{p}{(}\PY{n}{x1}\PY{p}{,}\PY{n}{psi1}\PY{p}{(}\PY{n}{x1}\PY{p}{,}\PY{l+m+mi}{3}\PY{p}{)}\PY{p}{)}
\PY{n}{plt}\PY{o}{.}\PY{n}{plot}\PY{p}{(}\PY{n}{x3}\PY{p}{,}\PY{n}{psi3}\PY{p}{(}\PY{n}{x3}\PY{p}{,}\PY{l+m+mi}{3}\PY{p}{)}\PY{p}{)}
\PY{n}{plt}\PY{o}{.}\PY{n}{plot}\PY{p}{(}\PY{n}{x2}\PY{p}{,}\PY{n}{psi2}\PY{p}{(}\PY{n}{x2}\PY{p}{,}\PY{l+m+mi}{3}\PY{p}{)}\PY{p}{)}
\end{Verbatim}
\end{tcolorbox}

            \begin{tcolorbox}[breakable, size=fbox, boxrule=.5pt, pad at break*=1mm, opacityfill=0]
\prompt{Out}{outcolor}{411}{\boxspacing}
\begin{Verbatim}[commandchars=\\\{\}]
[<matplotlib.lines.Line2D at 0x23c78a90>]
\end{Verbatim}
\end{tcolorbox}
        
    \begin{center}
    \adjustimage{max size={0.9\linewidth}{0.9\paperheight}}{output_11_1.png}
    \end{center}
    { \hspace*{\fill} \\}
    
    \hypertarget{densidades-de-probabilidad}{%
\subsection{densidades de
probabilidad}\label{densidades-de-probabilidad}}

    \begin{tcolorbox}[breakable, size=fbox, boxrule=1pt, pad at break*=1mm,colback=cellbackground, colframe=cellborder]
\prompt{In}{incolor}{415}{\boxspacing}
\begin{Verbatim}[commandchars=\\\{\}]
\PY{n}{fig}\PY{p}{,} \PY{n}{axs}\PY{o}{=}\PY{n}{plt}\PY{o}{.}\PY{n}{subplots}\PY{p}{(}\PY{l+m+mi}{2}\PY{p}{,}\PY{l+m+mi}{1}\PY{p}{)}
\PY{c+c1}{\PYZsh{}para n=0}
\PY{n}{axs}\PY{p}{[}\PY{l+m+mi}{0}\PY{p}{]}\PY{o}{.}\PY{n}{set\PYZus{}title}\PY{p}{(}\PY{l+s+s1}{\PYZsq{}}\PY{l+s+s1}{n=0}\PY{l+s+s1}{\PYZsq{}}\PY{p}{)}
\PY{n}{axs}\PY{p}{[}\PY{l+m+mi}{0}\PY{p}{]}\PY{o}{.}\PY{n}{plot}\PY{p}{(}\PY{n}{x1}\PY{p}{,}\PY{p}{(}\PY{n}{psi1}\PY{p}{(}\PY{n}{x1}\PY{p}{,}\PY{l+m+mi}{0}\PY{p}{)}\PY{p}{)}\PY{o}{*}\PY{o}{*}\PY{l+m+mi}{2}\PY{p}{)}
\PY{n}{axs}\PY{p}{[}\PY{l+m+mi}{0}\PY{p}{]}\PY{o}{.}\PY{n}{plot}\PY{p}{(}\PY{n}{x3}\PY{p}{,}\PY{p}{(}\PY{n}{psi3}\PY{p}{(}\PY{n}{x3}\PY{p}{,}\PY{l+m+mi}{0}\PY{p}{)}\PY{p}{)}\PY{o}{*}\PY{o}{*}\PY{l+m+mi}{2}\PY{p}{)}
\PY{n}{axs}\PY{p}{[}\PY{l+m+mi}{0}\PY{p}{]}\PY{o}{.}\PY{n}{plot}\PY{p}{(}\PY{n}{x2}\PY{p}{,}\PY{p}{(}\PY{n}{psi2}\PY{p}{(}\PY{n}{x2}\PY{p}{,}\PY{l+m+mi}{0}\PY{p}{)}\PY{p}{)}\PY{o}{*}\PY{o}{*}\PY{l+m+mi}{2}\PY{p}{)}

\PY{c+c1}{\PYZsh{}para n=1}
\PY{n}{axs}\PY{p}{[}\PY{l+m+mi}{1}\PY{p}{]}\PY{o}{.}\PY{n}{set\PYZus{}title}\PY{p}{(}\PY{l+s+s1}{\PYZsq{}}\PY{l+s+s1}{n=1}\PY{l+s+s1}{\PYZsq{}}\PY{p}{)}
\PY{n}{axs}\PY{p}{[}\PY{l+m+mi}{1}\PY{p}{]}\PY{o}{.}\PY{n}{plot}\PY{p}{(}\PY{n}{x1}\PY{p}{,}\PY{p}{(}\PY{n}{psi1}\PY{p}{(}\PY{n}{x1}\PY{p}{,}\PY{l+m+mi}{1}\PY{p}{)}\PY{p}{)}\PY{o}{*}\PY{o}{*}\PY{l+m+mi}{2}\PY{p}{)}
\PY{n}{axs}\PY{p}{[}\PY{l+m+mi}{1}\PY{p}{]}\PY{o}{.}\PY{n}{plot}\PY{p}{(}\PY{n}{x3}\PY{p}{,}\PY{p}{(}\PY{n}{psi3}\PY{p}{(}\PY{n}{x3}\PY{p}{,}\PY{l+m+mi}{1}\PY{p}{)}\PY{p}{)}\PY{o}{*}\PY{o}{*}\PY{l+m+mi}{2}\PY{p}{)}
\PY{n}{axs}\PY{p}{[}\PY{l+m+mi}{1}\PY{p}{]}\PY{o}{.}\PY{n}{plot}\PY{p}{(}\PY{n}{x2}\PY{p}{,}\PY{p}{(}\PY{n}{psi2}\PY{p}{(}\PY{n}{x2}\PY{p}{,}\PY{l+m+mi}{1}\PY{p}{)}\PY{p}{)}\PY{o}{*}\PY{o}{*}\PY{l+m+mi}{2}\PY{p}{)}
\PY{n}{fig}\PY{o}{.}\PY{n}{tight\PYZus{}layout}\PY{p}{(}\PY{p}{)}
\end{Verbatim}
\end{tcolorbox}

    \begin{center}
    \adjustimage{max size={0.9\linewidth}{0.9\paperheight}}{output_13_0.png}
    \end{center}
    { \hspace*{\fill} \\}
    
    \begin{tcolorbox}[breakable, size=fbox, boxrule=1pt, pad at break*=1mm,colback=cellbackground, colframe=cellborder]
\prompt{In}{incolor}{414}{\boxspacing}
\begin{Verbatim}[commandchars=\\\{\}]
\PY{n}{fig}\PY{p}{,} \PY{n}{axs}\PY{o}{=}\PY{n}{plt}\PY{o}{.}\PY{n}{subplots}\PY{p}{(}\PY{l+m+mi}{2}\PY{p}{,}\PY{l+m+mi}{1}\PY{p}{)}
\PY{n}{axs}\PY{p}{[}\PY{l+m+mi}{0}\PY{p}{]}\PY{o}{.}\PY{n}{set\PYZus{}title}\PY{p}{(}\PY{l+s+s1}{\PYZsq{}}\PY{l+s+s1}{n=2}\PY{l+s+s1}{\PYZsq{}}\PY{p}{)}
\PY{n}{axs}\PY{p}{[}\PY{l+m+mi}{0}\PY{p}{]}\PY{o}{.}\PY{n}{plot}\PY{p}{(}\PY{n}{x1}\PY{p}{,}\PY{p}{(}\PY{n}{psi1}\PY{p}{(}\PY{n}{x1}\PY{p}{,}\PY{l+m+mi}{2}\PY{p}{)}\PY{p}{)}\PY{o}{*}\PY{o}{*}\PY{l+m+mi}{2}\PY{p}{)}
\PY{n}{axs}\PY{p}{[}\PY{l+m+mi}{0}\PY{p}{]}\PY{o}{.}\PY{n}{plot}\PY{p}{(}\PY{n}{x3}\PY{p}{,}\PY{p}{(}\PY{n}{psi3}\PY{p}{(}\PY{n}{x3}\PY{p}{,}\PY{l+m+mi}{2}\PY{p}{)}\PY{p}{)}\PY{o}{*}\PY{o}{*}\PY{l+m+mi}{2}\PY{p}{)}
\PY{n}{axs}\PY{p}{[}\PY{l+m+mi}{0}\PY{p}{]}\PY{o}{.}\PY{n}{plot}\PY{p}{(}\PY{n}{x2}\PY{p}{,}\PY{p}{(}\PY{n}{psi2}\PY{p}{(}\PY{n}{x2}\PY{p}{,}\PY{l+m+mi}{2}\PY{p}{)}\PY{p}{)}\PY{o}{*}\PY{o}{*}\PY{l+m+mi}{2}\PY{p}{)}

\PY{n}{axs}\PY{p}{[}\PY{l+m+mi}{1}\PY{p}{]}\PY{o}{.}\PY{n}{set\PYZus{}title}\PY{p}{(}\PY{l+s+s1}{\PYZsq{}}\PY{l+s+s1}{n=3}\PY{l+s+s1}{\PYZsq{}}\PY{p}{)}
\PY{n}{axs}\PY{p}{[}\PY{l+m+mi}{1}\PY{p}{]}\PY{o}{.}\PY{n}{plot}\PY{p}{(}\PY{n}{x1}\PY{p}{,}\PY{p}{(}\PY{n}{psi1}\PY{p}{(}\PY{n}{x1}\PY{p}{,}\PY{l+m+mi}{3}\PY{p}{)}\PY{p}{)}\PY{o}{*}\PY{o}{*}\PY{l+m+mi}{2}\PY{p}{)}
\PY{n}{axs}\PY{p}{[}\PY{l+m+mi}{1}\PY{p}{]}\PY{o}{.}\PY{n}{plot}\PY{p}{(}\PY{n}{x3}\PY{p}{,}\PY{p}{(}\PY{n}{psi3}\PY{p}{(}\PY{n}{x3}\PY{p}{,}\PY{l+m+mi}{3}\PY{p}{)}\PY{p}{)}\PY{o}{*}\PY{o}{*}\PY{l+m+mi}{2}\PY{p}{)}
\PY{n}{axs}\PY{p}{[}\PY{l+m+mi}{1}\PY{p}{]}\PY{o}{.}\PY{n}{plot}\PY{p}{(}\PY{n}{x2}\PY{p}{,}\PY{p}{(}\PY{n}{psi2}\PY{p}{(}\PY{n}{x2}\PY{p}{,}\PY{l+m+mi}{3}\PY{p}{)}\PY{p}{)}\PY{o}{*}\PY{o}{*}\PY{l+m+mi}{2}\PY{p}{)}
\PY{n}{fig}\PY{o}{.}\PY{n}{tight\PYZus{}layout}\PY{p}{(}\PY{p}{)}
\end{Verbatim}
\end{tcolorbox}

    \begin{center}
    \adjustimage{max size={0.9\linewidth}{0.9\paperheight}}{output_14_0.png}
    \end{center}
    { \hspace*{\fill} \\}
    
    estuve revisando en la literarura y las graficas obtenidas no coinciden
con la caja cuantica, y no se porque

    \hypertarget{intento-de-resolver-computacionalmente-con-solve_bvp}{%
\subsection{intento de resolver computacionalmente con
solve\_bvp}\label{intento-de-resolver-computacionalmente-con-solve_bvp}}

    \begin{tcolorbox}[breakable, size=fbox, boxrule=1pt, pad at break*=1mm,colback=cellbackground, colframe=cellborder]
\prompt{In}{incolor}{418}{\boxspacing}
\begin{Verbatim}[commandchars=\\\{\}]
\PY{c+c1}{\PYZsh{} definimos la funcion a integrar }
\PY{k}{def} \PY{n+nf}{fun}\PY{p}{(}\PY{n}{x}\PY{p}{,} \PY{n}{y}\PY{p}{,} \PY{n}{p}\PY{p}{)}\PY{p}{:}
    \PY{n}{sk} \PY{o}{=} \PY{n}{p}\PY{p}{[}\PY{l+m+mi}{0}\PY{p}{]}
    \PY{k}{return} \PY{n}{np}\PY{o}{.}\PY{n}{vstack}\PY{p}{(}\PY{p}{(}\PY{n}{y}\PY{p}{[}\PY{l+m+mi}{1}\PY{p}{]}\PY{p}{,} \PY{o}{\PYZhy{}}\PY{n}{sk} \PY{o}{*} \PY{n}{y}\PY{p}{[}\PY{l+m+mi}{0}\PY{p}{]}\PY{p}{)}\PY{p}{)}
\PY{c+c1}{\PYZsh{}\PYZsh{}condiciones de frontera }

\PY{k}{def} \PY{n+nf}{bc}\PY{p}{(}\PY{n}{ya}\PY{p}{,} \PY{n}{yb}\PY{p}{,} \PY{n}{p}\PY{p}{)}\PY{p}{:}
    \PY{n}{k} \PY{o}{=} \PY{n}{p}\PY{p}{[}\PY{l+m+mi}{0}\PY{p}{]}
    \PY{k}{return} \PY{n}{np}\PY{o}{.}\PY{n}{array}\PY{p}{(}\PY{p}{[}\PY{n}{ya}\PY{p}{[}\PY{l+m+mi}{0}\PY{p}{]}\PY{o}{\PYZhy{}}\PY{n}{psi1}\PY{p}{(}\PY{o}{\PYZhy{}}\PY{n}{a}\PY{o}{/}\PY{l+m+mi}{2}\PY{p}{,}\PY{l+m+mi}{0}\PY{p}{)}\PY{p}{,} \PY{n}{yb}\PY{p}{[}\PY{l+m+mi}{0}\PY{p}{]}\PY{o}{\PYZhy{}}\PY{n}{psi3}\PY{p}{(}\PY{n}{a}\PY{o}{/}\PY{l+m+mi}{2}\PY{p}{,}\PY{l+m+mi}{0}\PY{p}{)}\PY{p}{,} \PY{n}{ya}\PY{p}{[}\PY{l+m+mi}{1}\PY{p}{]}\PY{o}{\PYZhy{}}\PY{n}{k} \PY{p}{]}\PY{p}{)}
    \PY{c+c1}{\PYZsh{} ya es la condicion para el lado izquierdo de la caja }

\PY{c+c1}{\PYZsh{}generamos los arreglos que van a contener las soluciones}
\PY{n}{x} \PY{o}{=} \PY{n}{np}\PY{o}{.}\PY{n}{linspace}\PY{p}{(}\PY{o}{\PYZhy{}}\PY{n}{a}\PY{o}{/}\PY{l+m+mi}{2}\PY{p}{,} \PY{n}{a}\PY{o}{/}\PY{l+m+mi}{2}\PY{p}{,} \PY{l+m+mi}{5}\PY{p}{)}
\PY{n}{y} \PY{o}{=} \PY{n}{np}\PY{o}{.}\PY{n}{zeros}\PY{p}{(}\PY{p}{(}\PY{l+m+mi}{2}\PY{p}{,} \PY{n}{x}\PY{o}{.}\PY{n}{size}\PY{p}{)}\PY{p}{)}
\PY{n}{y}\PY{p}{[}\PY{l+m+mi}{0}\PY{p}{,} \PY{l+m+mi}{1}\PY{p}{]} \PY{o}{=} \PY{l+m+mi}{100}
\PY{n}{y}\PY{p}{[}\PY{l+m+mi}{0}\PY{p}{,} \PY{l+m+mi}{3}\PY{p}{]} \PY{o}{=} \PY{l+m+mi}{0}

\PY{n}{sol} \PY{o}{=} \PY{n}{solve\PYZus{}bvp}\PY{p}{(}\PY{n}{fun}\PY{p}{,} \PY{n}{bc}\PY{p}{,} \PY{n}{x}\PY{p}{,} \PY{n}{y}\PY{p}{,} \PY{n}{p}\PY{o}{=}\PY{p}{[}\PY{l+m+mi}{2}\PY{o}{*}\PY{n}{E}\PY{p}{]}\PY{p}{)}
\end{Verbatim}
\end{tcolorbox}

    \begin{tcolorbox}[breakable, size=fbox, boxrule=1pt, pad at break*=1mm,colback=cellbackground, colframe=cellborder]
\prompt{In}{incolor}{421}{\boxspacing}
\begin{Verbatim}[commandchars=\\\{\}]
\PY{c+c1}{\PYZsh{} grafica de la funcion para n=0}

\PY{n}{x\PYZus{}plot} \PY{o}{=} \PY{n}{np}\PY{o}{.}\PY{n}{linspace}\PY{p}{(}\PY{o}{\PYZhy{}}\PY{n}{a}\PY{o}{/}\PY{l+m+mi}{2}\PY{p}{,} \PY{n}{a}\PY{o}{/}\PY{l+m+mi}{2}\PY{p}{,} \PY{l+m+mi}{100}\PY{p}{)}
\PY{n}{y\PYZus{}plot} \PY{o}{=} \PY{n}{sol}\PY{o}{.}\PY{n}{sol}\PY{p}{(}\PY{n}{x\PYZus{}plot}\PY{p}{)}\PY{p}{[}\PY{l+m+mi}{0}\PY{p}{]}
\PY{n}{plt}\PY{o}{.}\PY{n}{plot}\PY{p}{(}\PY{n}{x\PYZus{}plot}\PY{p}{,} \PY{n}{y\PYZus{}plot}\PY{p}{)}
\PY{n}{plt}\PY{o}{.}\PY{n}{xlabel}\PY{p}{(}\PY{l+s+s2}{\PYZdq{}}\PY{l+s+s2}{x}\PY{l+s+s2}{\PYZdq{}}\PY{p}{)}
\PY{n}{plt}\PY{o}{.}\PY{n}{ylabel}\PY{p}{(}\PY{l+s+s2}{\PYZdq{}}\PY{l+s+s2}{y}\PY{l+s+s2}{\PYZdq{}}\PY{p}{)}
\PY{n}{plt}\PY{o}{.}\PY{n}{plot}\PY{p}{(}\PY{n}{x1}\PY{p}{,}\PY{n}{psi1}\PY{p}{(}\PY{n}{x1}\PY{p}{,}\PY{l+m+mi}{0}\PY{p}{)}\PY{p}{)}
\PY{n}{plt}\PY{o}{.}\PY{n}{plot}\PY{p}{(}\PY{n}{x3}\PY{p}{,}\PY{n}{psi3}\PY{p}{(}\PY{n}{x3}\PY{p}{,}\PY{l+m+mi}{0}\PY{p}{)}\PY{p}{)}
\end{Verbatim}
\end{tcolorbox}

            \begin{tcolorbox}[breakable, size=fbox, boxrule=.5pt, pad at break*=1mm, opacityfill=0]
\prompt{Out}{outcolor}{421}{\boxspacing}
\begin{Verbatim}[commandchars=\\\{\}]
[<matplotlib.lines.Line2D at 0x1b378f40>]
\end{Verbatim}
\end{tcolorbox}
        
    \begin{center}
    \adjustimage{max size={0.9\linewidth}{0.9\paperheight}}{output_18_1.png}
    \end{center}
    { \hspace*{\fill} \\}
    
    La grafica no corresponde a lo que dio realizando el proceso
teoricamente, por tanto descarte completamente la forma en la que estaba
solucionando el metodo

    \begin{tcolorbox}[breakable, size=fbox, boxrule=1pt, pad at break*=1mm,colback=cellbackground, colframe=cellborder]
\prompt{In}{incolor}{ }{\boxspacing}
\begin{Verbatim}[commandchars=\\\{\}]

\end{Verbatim}
\end{tcolorbox}


    % Add a bibliography block to the postdoc
    
    
    
\end{document}
